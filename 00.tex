\maketitle

\setlength{\epigraphwidth}{0.6\textwidth}
\epigraph{\it Mathematics is a process of staring hard enough with enough perseverance at the fog of muddle and confusion to eventually break through to improved clarity. I’m happy when I can admit, at least to myself, that my thinking is muddled, and I try to overcome the embarrassment that I might reveal ignorance or confusion. }{William Thurston}


\section*{How to use these notes.}
\label{sec:intro}
\addcontentsline{toc}{section}{\nameref{sec:intro}}

Linear algebra is the study of linearity. The main players of this story are the vectors, but instead of studying vectors individually, we study collections of vectors, called {\bf vector spaces}, and maps between vector spaces, called {\bf linear transformations}.

Because linearity is such a simple condition (Section \ref{section:VectorSpaces}), linear algebra is one of the most widely applied branches of mathematics.\\\\

These notes provide a very, very brief introduction to linear algebra.
You'll learn enough to be able to continue the study of the subject on your own.

Each day you will be given a set of problems to solve in class.
You should attempt as many problems as you can, and it is ok to not finish all the problems before the next class.
However, you should make an honest attempt to read and understand the various definitions and theorems.\footnote{These notes are likely to contain \sout{a few} several \sout{typos} mistakes. Please do tell if you find any, thanks in advance.}
