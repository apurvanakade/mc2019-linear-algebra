\section*{Summary of Section 1}
\begin{itemize}
  \item In higher dimensions,
      \begin{enumerate}
        \item a line passing through the origin is described as
        \begin{equation*}
          L = \{ c \vec{v} : c \in \bbr \}
        \end{equation*}
        \item a plane passing through the origin is described as
        \begin{equation*}
          P = \{ c_1 \vec{v}_1 + c_2 \vec{v}_2 : c_1, c_2 \in \bbr \}
        \end{equation*}
        when $\vec{v}_1$ is not a scalar multiple of $\vec{v}_2$.
      \end{enumerate}
    We generalize this using the notion of $\spn$. $L$ is the span of the set $\{\vec{v}\}$ and $P$ is the span of the set $\{ \vec{v}_1, \vec{v}_2 \}$.

    \item More generally, for any subset $\cals \subset \bbr^n$
    \begin{align*}
      \spn(\cals) = \{ c_1 \vec{v}_1 + \dots + c_k \vec{v}_k \quad : \quad
      & c_1, \dots, c_k \mbox{ are real numbers}, \\
      & \vec{v}_1, \dots, \vec{v}_k \mbox{ are vectors in } S \}
    \end{align*}
    defines a subspace of $\bbr^n$. A subspace is a subset of $\bbr^n$ which is closed under scalar multiplication and addition.

    However, there might be reduncacies in $\cals$. For example,
    \begin{align*}
      \spn(\vec{v}_1) &= \spn(\vec{v}_1, \vec{v}_2)
      && \mbox{ if $\vec{v}_1$ is a scalar multiple of $\vec{v}_2$} \\
      \spn(\vec{e}_1 - \vec{e}_2, \vec{e}_2 - \vec{e}_3, \vec{e}_3 - \vec{e}_1) &= \spn(\vec{e}_1 - \vec{e}_2, \vec{e}_2 - \vec{e}_3)
    \end{align*}

    \item This is ``corrected'' by linear independence (Definition \ref{def:linearIndependence}). For a subspace $V$ of $\bbr^n$, a set $\calb \subseteq V$ is said to be a \emph{basis} of $V$ if $\spn(\calb) = V$ and ${\calb}$ is linearly independent.

    In the above examples, $L$ has basis $\calb = \{ \vec{v} \}$ and $P$ has basis $\{ \vec{v}_1, \vec{v}_2 \}$.

    \item A basis allows us to define coordinates for a vector space.
    \begin{enumerate}
      \item For $L$ as above, with $\calb = \{ \vec{v} \}$
      \begin{equation*}
        [c\vec{v}]_{\calb} = [ c ]
      \end{equation*}
      \item For $P$ as above, with $\calb = \{ \vec{v}_1, \vec{v}_2\}$
      \begin{equation*}
        [c_1 \vec{v}_1 + c_2 \vec{v}_2]_{\calb} = \begin{bmatrix}
          c_1 \\ c_2
        \end{bmatrix}
      \end{equation*}
      \item More generally, if $V$ has basis $\calb = \{ \vec{v}_1, \dots, \vec{v}_k\}$ then
      \begin{equation*}
        [c_1 \vec{v}_1 + \dots + c_k \vec{v}_k]_{\calb} = \begin{bmatrix}
          c_1 \\ \vdots \\c_k
        \end{bmatrix}
      \end{equation*}
    \end{enumerate}
\end{itemize}

\begin{figure}
  \centering
  \begin{tikzpicture}[
     >=latex,
     % font=\footnotesize,
     x={(1cm, 1cm)},
     y={(1cm, -.5cm)},
   ]
     \def\xmin{-2}
     \def\xmax{4}
     \def\ymin{-2}
     \def\ymax{4}
     \draw[thin, dashed]
       \foreach \x in {\xmin, ..., \xmax} {
         (\x, \ymin) -- (\x, \ymax)
       }
       \foreach \y in {\ymin, ..., \ymax} {
         (\xmin, \y) -- (\xmax, \y)
       }
     ;
     \draw[<->, very thick]
       (1, 0) node[above] {$\vec{v}_1$}
       -- (0, 0)
       -- (0, 1) node[below] {$\vec{v}_2$}
     ;
     \draw[->]
       (0, 0)
       -- (2, 3) node[above] {$\begin{bmatrix} 2 \\ 3 \end{bmatrix}$}
     ;
   \end{tikzpicture}
  \caption{Generalized coordinates on a plane. In the basis $\calb=\{ \vec{v}_1, \vec{v}_2 \}$, the vector $\vec{v} = 2 \vec{v}_1 + 3 \vec{v}_2$ has coordinates $[\vec{v}]_{\calb} = \protect\begin{bmatrix} 2 \\ 3 \protect\end{bmatrix}$.}
\end{figure}
